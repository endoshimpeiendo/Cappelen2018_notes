%%%%%%%%%%%%%%%%%%%%%%%%%%%%%%%%%%%%%%%%%
% Article Notes
% LaTeX Template
% Version 1.0 (1/10/15)
%
% This template has been downloaded from:
% http://www.LaTeXTemplates.com
%
% Authors:
% Vel (vel@latextemplates.com)
% Christopher Eliot (christopher.eliot@hofstra.edu)
% Anthony Dardis (anthony.dardis@hofstra.edu)
%
% License:
% CC BY-NC-SA 3.0 (http://creativecommons.org/licenses/by-nc-sa/3.0/)
%
%%%%%%%%%%%%%%%%%%%%%%%%%%%%%%%%%%%%%%%%%

%----------------------------------------------------------------------------------------
%	PACKAGES AND OTHER DOCUMENT CONFIGURATIONS
%----------------------------------------------------------------------------------------

\documentclass[
10pt, % Default font size is 10pt, can alternatively be 11pt or 12pt
a4paper, % Alternatively letterpaper for US letter
twocolumn, % Alternatively onecolumn
landscape % Alternatively portrait
]{article}

%%%%%%%%%%%%%%%%%%%%%%%%%%%%%%%%%%%%%%%%%
% Article Notes
% Structure Specification File
% Version 1.0 (1/10/15)
%
% This file has been downloaded from:
% http://www.LaTeXTemplates.com
%
% Authors:
% Vel (vel@latextemplates.com)
% Christopher Eliot (christopher.eliot@hofstra.edu)
% Anthony Dardis (anthony.dardis@hofstra.edu)
%
% License:
% CC BY-NC-SA 3.0 (http://creativecommons.org/licenses/by-nc-sa/3.0/)
%
%%%%%%%%%%%%%%%%%%%%%%%%%%%%%%%%%%%%%%%%%

%----------------------------------------------------------------------------------------
%	REQUIRED PACKAGES
%----------------------------------------------------------------------------------------

\usepackage[includeheadfoot,columnsep=2cm, left=1in, right=1in, top=.5in, bottom=.5in]{geometry} % Margins

\usepackage[T1]{fontenc} % For international characters
\usepackage{XCharter} % XCharter as the main font

\usepackage{natbib} % Use natbib to manage the reference
\bibliographystyle{apalike} % Citation style

\usepackage[english]{babel} % Use english by default

%----------------------------------------------------------------------------------------
%	CUSTOM COMMANDS
%----------------------------------------------------------------------------------------

\newcommand{\articletitle}[1]{\renewcommand{\articletitle}{#1}} % Define a command for storing the article title
\newcommand{\articlecitation}[1]{\renewcommand{\articlecitation}{#1}} % Define a command for storing the article citation
\newcommand{\doctitle}{\articlecitation\ --- ``\articletitle''} % Define a command to store the article information as it will appear in the title and header

\newcommand{\datenotesstarted}[1]{\renewcommand{\datenotesstarted}{#1}} % Define a command to store the date when notes were first made
\newcommand{\docdate}[1]{\renewcommand{\docdate}{#1}} % Define a command to store the date line in the title

\newcommand{\docauthor}[1]{\renewcommand{\docauthor}{#1}} % Define a command for storing the article notes author

% Define a command for the structure of the document title
\newcommand{\printtitle}{
\begin{center}
\textbf{\Large{\doctitle}}

\docdate

\docauthor
\end{center}
}

%----------------------------------------------------------------------------------------
%	STRUCTURE MODIFICATIONS
%----------------------------------------------------------------------------------------

\setlength{\parskip}{3pt} % Slightly increase spacing between paragraphs

% Uncomment to center section titles
%\usepackage{sectsty}
%\sectionfont{\centering}

% Uncomment for Roman numerals for section numbers
%\renewcommand\thesection{\Roman{section}}
 % Input the file specifying the document layout and structure
\usepackage{hyperref}


%----------------------------------------------------------------------------------------
%	ARTICLE INFORMATION
%----------------------------------------------------------------------------------------

\articletitle{Summary of ch.18 Conceptual Engineering without Bedrock and without Fixed Points} % The title of the article
\articlecitation{\cite{Cappelen2018FixingLanguage}} % The BibTeX citation key from your bibliography

\datenotesstarted{January 13, 2019} % The date when these notes were first made
\docdate{\datenotesstarted; rev. \today} % The date when the notes were lasted updated (automatically the current date)
\docauthor{Summarized and Commented by Shimpei Endo} % Your name

%----------------------------------------------------------------------------------------

\begin{document}

\pagestyle{myheadings} % Use custom headers
\markright{\doctitle} % Place the article information into the header

%----------------------------------------------------------------------------------------
%	PRINT ARTICLE INFORMATION
%----------------------------------------------------------------------------------------

\thispagestyle{plain} % Plain formatting on the first page

\printtitle % Print the title

%----------------------------------------------------------------------------------------
%	ARTICLE NOTES
%----------------------------------------------------------------------------------------

\section*{In a nutshell... }
This section defends the universal nature of conceptual engineering: anything can be
a target of conceptual engineering.
Cappelen disagrees with two arguments by Chalmers and by Eklund.

\noindent \textbf{Keywords:} Chalmers, Eklund

\section*{18.1 The Limits of Conceptual Engineering?}
\paragraph{You can conceptual-engineer anything. }
Is there any term so fundamental that we cannot conceptual-engineer?
Chalmers and Eklund independenty hold their view supporting that
``there are specific features of specific meanings that make
the idea of evolution problematic'' [p. 194].
Cappelen rejects them.
His recommended hypothesis is: ``\emph{everything is in flux -- that all representational devices
can be revised, and there's no natural end point to conceptual engineering}''[p. 194, his emphasis].

\section*{18.2 Chalmers on Bedrock Concepts}
Chalmers is a supporter of ``bedrock concepts'' \textit{a la} Cappelen, ``expressions that cannot be engineered'' [p. 194].

\paragraph{Chalmers' tenative list of bedrock concepts.}
Chalmers' candidates include: phenomenal concepts, normative concepts, logical or mathematical concepts,
spatiotemporal concepts, indexical concepts, cocepts of explanation; furthermore,
secondary qualities, intentional states, freedom.
\cite[p.552]{Chalmers2011}

\paragraph{How to find? Asymmetrical structure among the space of disputes.}

\paragraph{Cappelen does not agree.}
[p. 195]

\section*{18.3 Eklund on Normative Limits}
\paragraph{Eklund: we cannot engineer normative concepts.}
According to Eklund, we cannot allow the very normative concepts (e.g. `right' `good' or `ought') being used in the process of conceptual engineering to be changed.
``[O]ur thinnest normative concepts are irreplaceable''[p. 197].

\paragraph{Cappelen's starting lines.}
Cappelen hesitatedly offers three quick counterarguments towards Eklund.

\section*{Comments by Endo}
%----------------------------------------------------------------------------------------
%	BIBLIOGRAPHY
%----------------------------------------------------------------------------------------

%\renewcommand{\refname}{Reference} % Change the default bibliography title

\bibliography{Mendeley} % Input your bibliography file

%----------------------------------------------------------------------------------------

\end{document}
