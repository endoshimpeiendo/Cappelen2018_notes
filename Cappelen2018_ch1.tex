%%%%%%%%%%%%%%%%%%%%%%%%%%%%%%%%%%%%%%%%%
% Article Notes
% LaTeX Template
% Version 1.0 (1/10/15)
%
% This template has been downloaded from:
% http://www.LaTeXTemplates.com
%n
% Authors:
% Vel (vel@latextemplates.com)
% Christopher Eliot (christopher.eliot@hofstra.edu)
% Anthony Dardis (anthony.dardis@hofstra.edu)
%
% License:
% CC BY-NC-SA 3.0 (http://creativecommons.org/licenses/by-nc-sa/3.0/)
%
%%%%%%%%%%%%%%%%%%%%%%%%%%%%%%%%%%%%%%%%%

%----------------------------------------------------------------------------------------
%	PACKAGES AND OTHER DOCUMENT CONFIGURATIONS
%----------------------------------------------------------------------------------------

\documentclass[
10pt, % Default font size is 10pt, can alternatively be 11pt or 12pt
a4paper, % Alternatively letterpaper for US letter
twocolumn, % Alternatively onecolumn
landscape % Alternatively portrait
]{article}

%%%%%%%%%%%%%%%%%%%%%%%%%%%%%%%%%%%%%%%%%
% Article Notes
% Structure Specification File
% Version 1.0 (1/10/15)
%
% This file has been downloaded from:
% http://www.LaTeXTemplates.com
%
% Authors:
% Vel (vel@latextemplates.com)
% Christopher Eliot (christopher.eliot@hofstra.edu)
% Anthony Dardis (anthony.dardis@hofstra.edu)
%
% License:
% CC BY-NC-SA 3.0 (http://creativecommons.org/licenses/by-nc-sa/3.0/)
%
%%%%%%%%%%%%%%%%%%%%%%%%%%%%%%%%%%%%%%%%%

%----------------------------------------------------------------------------------------
%	REQUIRED PACKAGES
%----------------------------------------------------------------------------------------

\usepackage[includeheadfoot,columnsep=2cm, left=1in, right=1in, top=.5in, bottom=.5in]{geometry} % Margins

\usepackage[T1]{fontenc} % For international characters
\usepackage{XCharter} % XCharter as the main font

\usepackage{natbib} % Use natbib to manage the reference
\bibliographystyle{apalike} % Citation style

\usepackage[english]{babel} % Use english by default

%----------------------------------------------------------------------------------------
%	CUSTOM COMMANDS
%----------------------------------------------------------------------------------------

\newcommand{\articletitle}[1]{\renewcommand{\articletitle}{#1}} % Define a command for storing the article title
\newcommand{\articlecitation}[1]{\renewcommand{\articlecitation}{#1}} % Define a command for storing the article citation
\newcommand{\doctitle}{\articlecitation\ --- ``\articletitle''} % Define a command to store the article information as it will appear in the title and header

\newcommand{\datenotesstarted}[1]{\renewcommand{\datenotesstarted}{#1}} % Define a command to store the date when notes were first made
\newcommand{\docdate}[1]{\renewcommand{\docdate}{#1}} % Define a command to store the date line in the title

\newcommand{\docauthor}[1]{\renewcommand{\docauthor}{#1}} % Define a command for storing the article notes author

% Define a command for the structure of the document title
\newcommand{\printtitle}{
\begin{center}
\textbf{\Large{\doctitle}}

\docdate

\docauthor
\end{center}
}

%----------------------------------------------------------------------------------------
%	STRUCTURE MODIFICATIONS
%----------------------------------------------------------------------------------------

\setlength{\parskip}{3pt} % Slightly increase spacing between paragraphs

% Uncomment to center section titles
%\usepackage{sectsty}
%\sectionfont{\centering}

% Uncomment for Roman numerals for section numbers
%\renewcommand\thesection{\Roman{section}}
 % Input the file specifying the document layout and structure
\usepackage{hyperref}


%----------------------------------------------------------------------------------------
%	ARTICLE INFORMATION
%----------------------------------------------------------------------------------------

\articletitle{Note on Ch.1 Introduction of  \textit{Fixing Language: An Essay for Conceptual Engineering}} % The title of the article
\articlecitation{\cite{Cappelen2018FixingLanguage}} % The BibTeX citation key from your bibliography

\datenotesstarted{November 30, 2018} % The date when these notes were first made
\docdate{\datenotesstarted; rev. \today} % The date when the notes were lasted updated (automatically the current date)
\docauthor{Summarized ands commented by Shimpei Endo} % Your name

%----------------------------------------------------------------------------------------

\begin{document}

\pagestyle{myheadings} % Use custom headers
\markright{\doctitle} % Place the article information into the header

%----------------------------------------------------------------------------------------
%	PRINT ARTICLE INFORMATION
%----------------------------------------------------------------------------------------

\thispagestyle{plain} % Plain formatting on the first page

\printtitle % Print the title

%----------------------------------------------------------------------------------------
%	ARTICLE NOTES
%----------------------------------------------------------------------------------------

\section*{In a nutshell... }
This introduction outlines what this book is about.
The very term \emph{conceptual engineering} is explained as the process(es) of assessing and improving our representational devices (\emph{concepts}) with observing several other terminologies for the seemingly same/similar practice(1.1).
Cappelen characterizes his own position as representational skepticist with comparison to representational complacent, who swallow the given concepts without questioning their given concepts (1.2).
Cappelen also presents the construction(1.1)  and central topics through this book (1.3).

\paragraph{Keywords:}
conceptual engineering,
representational devices,
metasemantics,
Austerity Framework,
representational skepticism,
externalism,
continuity of inquiry

%------------------------------------------------
\section*{1.1 Introduction}
\paragraph{Representational devices=concepts!}
This book is about the process of assesing and improving our \emph{representational devices}, which Capellen calls \emph{concepts}.

\paragraph{How have we called it?}
 \cite{Blackburn1999} and \cite{Eklund2010} already call their job as \emph{conceptual engineering}.
 \cite{Haslanger2012} labells her own contributes as \emph{ameliorative projects} or \emph{analytical projects}.
The term \emph{revisionary project} is adopted by \cite{Railton1993} and \cite{Scharp2013}.
Carnap's \emph{explication} is a variant of projects which this book is about.
Burgess and Plunkett prefer \emph{conceptual ethics}.

\paragraph{Conceptual engineering is not about concept nor is not engineering!}
After provided the list of similar practices, interestingly (and disapointly for some?), Cappelen confesses:
``It's important that readers don't take that name as a description: on the view I defend in this book, the project isn't about concepts and there isn't really any engeneering.'' (p.4)

\paragraph{The construction of this book}
This book has five parts. Cappelen suggests the Austerity Framework in the middle parts: II thought IV.

\begin{description}
\item[Part I.] Settle down the targets.
Chapter 2 offers examples. Chpater 3 and 4 discuss more general issues.
\item[Part II.] Build a metasemantic ground for conceptual engeneering.
\item[Part III.] Argue the limit of engeneering.
In particular, Cappelen tackles objections sayning that conceptual engeneering is just changing the subject (cf. \emph{continuity}).
The objections are constructed in chapter 9 and the following chapters (10 and 11) response to it.
\item[Part IV.] Complete Austery Framework.
\item[Part V.] Compare with other approaches.
This part, for instance, considers metasematnic negotiation, conceptual function, and Chalmers' elimination.
\end{description}


\section*{1.2 A Heuristic: representational complacency vs. representational skepticism}

People from different backgrounds other than philosophy can join conceptual engeneering. Cappelen adopts a heuristic for dividing (roughly) people into two groups: the representationally \emph{complacent} and the representationallly \emph{skeptics}.
The complacent does not question concepts given to them.
Note that skeptics on object-level is possible to excute in complacent in meta-level.
The skeptics question concepts given to them and tries to improve them. The latter position is what conceptual engineers should belong to.

\paragraph{Look at himself.}
Cappelen himself is a sample of this skeptic tribe.
In fact, Cappelen has argued that concepts such as intuition and \textit{de se} are so defective terminologies that cause philosophical problems.
Cappelen mentions that skeptic attitude may also hold to what terminology/concept describes representationally skeptics themselves. An infinite regress? Or we should not talk about them at all because we cannot have a proper language to talk about? Cappelen does not have any conclusive discussion or promissing strategy. Cappelen has rather sees this book as a \emph{progress report}. 

\section*{1.3 Central themes of this book}
We will observe several examples in the next chapter 2. Notice that this book does not provide a detailed plan for improvement, which is out of the aim and scope of this monograph. Cappelen sees things more ``from above''.
To avoid losing track, Cappelen lists the six main themes of this book at the end of this introductiory chapter.

\begin{enumerate}
  \item \emph{A theory of metasemantics works at the foundation of a theory of conceptual engeneering.}
See part II. We have not reached any concensus of which metasemantic we should adopt.

  \item \emph{Conceptual engeneering is compatible with externalism.}
  A reasonable question towrads conceptual engineers asks how to keep consistency with \emph{externalism}.
  Given we the users themselves change our concepts being used, does not it entail \emph{internalism}, implying that concepts are all about our mental activities and it is us which determines concepts?
Part II, particularly chapter 6 defends externalims within the Austery Framework.

  \item \emph{In or out of our control?}
  Relating to the theme 2, Cappelen himself takes quite an unique position insisting ``not in our control.''
In a nutshell, ``being in control is overrated and for the most part an illusion anyway'' (p.8).

  \item \emph{No systematic theory!}
  Warning: this book is not a checklist nor a beginner's guide for successful conceptual engeneering. Instead, this book argues there is \emph{no} such a manual.
Revisionism works on \emph{meta-levels} too; the rules governing conceptual engeneering also get improved and revised as concepts in object-level do.

  \item \emph{Conceptual engeneering and continuity of inquiry.}
  See Part III.
Another possible concern is about continuity: can we still keep doing what we have done like arguing, (dis)agreeing on the same thing etc.?
Chapter 9, for example, re-constructs Strawson's worry, which belongs to this line, towrads Carnapian explication. And proceeding chapters will successfully respond it!

  \item \emph{Conceptual engeneering changes the world.}
  For example, when we improve our concept/word `race', we also improve or change race.
Conceptual amelioration (cf. Haslanger) is amelioration of the world.

\end{enumerate}

\section*{Comments by me.}
%----------------------------------------------------------------------------------------
%	BIBLIOGRAPHY
%----------------------------------------------------------------------------------------

\renewcommand{\refname}{Reference} % Change the default bibliography title

\bibliography{Mendeley} % Input your bibliography file

%----------------------------------------------------------------------------------------

\end{document}
