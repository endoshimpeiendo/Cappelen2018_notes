%%%%%%%%%%%%%%%%%%%%%%%%%%%%%%%%%%%%%%%%%
% Article Notes
% LaTeX Template
% Version 1.0 (1/10/15)
%
% This template has been downloaded from:
% http://www.LaTeXTemplates.com
%
% Authors:
% Vel (vel@latextemplates.com)
% Christopher Eliot (christopher.eliot@hofstra.edu)
% Anthony Dardis (anthony.dardis@hofstra.edu)
%
% License:
% CC BY-NC-SA 3.0 (http://creativecommons.org/licenses/by-nc-sa/3.0/)
%
%%%%%%%%%%%%%%%%%%%%%%%%%%%%%%%%%%%%%%%%%

%----------------------------------------------------------------------------------------
%	PACKAGES AND OTHER DOCUMENT CONFIGURATIONS
%----------------------------------------------------------------------------------------

\documentclass[
10pt, % Default font size is 10pt, can alternatively be 11pt or 12pt
a4paper, % Alternatively letterpaper for US letter
twocolumn, % Alternatively onecolumn
landscape % Alternatively portrait
]{article}

%%%%%%%%%%%%%%%%%%%%%%%%%%%%%%%%%%%%%%%%%
% Article Notes
% Structure Specification File
% Version 1.0 (1/10/15)
%
% This file has been downloaded from:
% http://www.LaTeXTemplates.com
%
% Authors:
% Vel (vel@latextemplates.com)
% Christopher Eliot (christopher.eliot@hofstra.edu)
% Anthony Dardis (anthony.dardis@hofstra.edu)
%
% License:
% CC BY-NC-SA 3.0 (http://creativecommons.org/licenses/by-nc-sa/3.0/)
%
%%%%%%%%%%%%%%%%%%%%%%%%%%%%%%%%%%%%%%%%%

%----------------------------------------------------------------------------------------
%	REQUIRED PACKAGES
%----------------------------------------------------------------------------------------

\usepackage[includeheadfoot,columnsep=2cm, left=1in, right=1in, top=.5in, bottom=.5in]{geometry} % Margins

\usepackage[T1]{fontenc} % For international characters
\usepackage{XCharter} % XCharter as the main font

\usepackage{natbib} % Use natbib to manage the reference
\bibliographystyle{apalike} % Citation style

\usepackage[english]{babel} % Use english by default

%----------------------------------------------------------------------------------------
%	CUSTOM COMMANDS
%----------------------------------------------------------------------------------------

\newcommand{\articletitle}[1]{\renewcommand{\articletitle}{#1}} % Define a command for storing the article title
\newcommand{\articlecitation}[1]{\renewcommand{\articlecitation}{#1}} % Define a command for storing the article citation
\newcommand{\doctitle}{\articlecitation\ --- ``\articletitle''} % Define a command to store the article information as it will appear in the title and header

\newcommand{\datenotesstarted}[1]{\renewcommand{\datenotesstarted}{#1}} % Define a command to store the date when notes were first made
\newcommand{\docdate}[1]{\renewcommand{\docdate}{#1}} % Define a command to store the date line in the title

\newcommand{\docauthor}[1]{\renewcommand{\docauthor}{#1}} % Define a command for storing the article notes author

% Define a command for the structure of the document title
\newcommand{\printtitle}{
\begin{center}
\textbf{\Large{\doctitle}}

\docdate

\docauthor
\end{center}
}

%----------------------------------------------------------------------------------------
%	STRUCTURE MODIFICATIONS
%----------------------------------------------------------------------------------------

\setlength{\parskip}{3pt} % Slightly increase spacing between paragraphs

% Uncomment to center section titles
%\usepackage{sectsty}
%\sectionfont{\centering}

% Uncomment for Roman numerals for section numbers
%\renewcommand\thesection{\Roman{section}}
 % Input the file specifying the document layout and structure
\usepackage{hyperref}


%----------------------------------------------------------------------------------------
%	ARTICLE INFORMATION
%----------------------------------------------------------------------------------------

\articletitle{Summary of ch.2 Conceptual Engineering in Philosophy and Beyond} % The title of the article
\articlecitation{\cite{Cappelen2018FixingLanguage}} % The BibTeX citation key from your bibliography

\datenotesstarted{December 6, 2018} % The date when these notes were first made
\docdate{\datenotesstarted; rev. \today} % The date when the notes were lasted updated (automatically the current date)
\docauthor{Summarized ands commented by Shimpei Endo} % Your name

%----------------------------------------------------------------------------------------

\begin{document}

\pagestyle{myheadings} % Use custom headers
\markright{\doctitle} % Place the article information into the header

%----------------------------------------------------------------------------------------
%	PRINT ARTICLE INFORMATION
%----------------------------------------------------------------------------------------

\thispagestyle{plain} % Plain formatting on the first page

\printtitle % Print the title

%----------------------------------------------------------------------------------------
%	ARTICLE NOTES
%----------------------------------------------------------------------------------------

\section*{In a nutshell... }
The second chapter (p.9-38) showcases actual practices of conceptual engineering.
The sources of examples vary from philosophical (section I) and nonphilosophical (esp. political and social) contexts (section II).
The final section III offers a general but tenative guideline or taxmonomy of conceptual engineering which at least covers examples given in this chapter.

\noindent \textbf{Keywords:} descriptive, normative, Carnap, Chalmers, Haslanger,

\section*{2.1 Conceptual Engineering in Philosophy}
\paragraph{Preliminary cautions.}
Before showing the particular examples in philosophy (2.1.1-2.1.10), Cappelen mentions three preliminary notes.

\begin{enumerate}
  \item
    Philosophy is \emph{not} special. Philosophy just offers nice examples of ``explicit and theoretically self-aware instances of conceptual engineering.'' (p. 9)
  \item
    There seemed to be no \emph{unified} topic (this explains why no book about conceptual engineering has not been published). Cappelen believes there is a unified (and even rich) one.
  \item
    Philosophical (and other) examples are \emph{data points}.
\end{enumerate}

\subsection*{2.1.1 Clark and Chalmers on `belief' and the extended mind}
The first example is an analysis of belief and the extended mind by Clark and Chalmers.
According to Clark and Chalmers,  `A believes that p' is understood as something is true even when A has access to p proposition which needs the essistance of various `external' devices.

Some would reply that their analysis objects that ``we do not use `belief' in that way''.
However, Clark and Chalmers do not describe but \emph{revise} our current concept of `belief'.

Observe the two key characteristics of Clark and Chalmers' suggestion.
First, their revision affects not only extension but also intension of the concept `belief'.
Second, they briefly justify the revision for being more useful, deeper, and more unified.

\subsection*{2.1.2 Explication: from Carnap and Quine to Gupta}

The second instance from philosphy is taken from Carnap-Quine (and Gupta).

Carnap insists that philosophers should work for \emph{explication}.
By his own words, ``[t]he task of explication consists in trans- forming a given more or less inexact concept into an exact one or, rather, in replacing the first by the second'' (Carnap 1950: 3).
In a way, Cappelen is a Carnapian for doing explication, i.e. ``the idea that we take a term that has a certain deficiency and then transform it into a better concept'' (p. 11)
Cappelen expands the area of explication into four dimensions: (relaxed) similarlity, usage, fruitfulness, and simplicity. Carnap only has ``inexactness''.
Quine said similar things.
Carnap also writes that there is \emph{no} unique right explication of any term.

\subsection*{2.1.3 Haslanger on amelioration in general and of gender and race terms in particular}
What Haslanger calls `ameliorative projects' is an example of conceptual engineering. Instead of describing our concepts or their extensions, an ameliorative project aims to answer the following questions:
\begin{quote}
  What is the point of having these concepts? What cognitive or practical task do they (or should they) enable us to accomplish? Are they effective tools to accomplish our (legitimate) purposes; if not, what concepts would serve these purposes better? (Haslanger 2000:33)
\end{quote}

Haslanger actually \emph{executed} amelioration (e.g. to revise the meaning of the word `man' and `woman').
Cappelen underlines the two things about her proposals.
First, the ameliorative proposal is \emph{revisionary}. Second, the ameliorative project is justified by political reasons. Her goal is to eliminate what she defines as women (which is different from female, according to Haslanger)!


\subsection*{2.1.4 Revisionism about moral language}
Most moral theories are \emph{revisionary}.
Their common structure is (a) argue that moral language is flawed then (b) propose how to improve it.

\subsubsection*{2.1.4.1 Railton's revisionism}
Railton hold a naturalistic view (i) seeing the moral (and more generaly philosophical) language as naturalistically unacceptable and (ii) leading to revise our modal language so that fits a naturalistic view.
Later in Part III, Cappelen deals with Railton's solution.

\subsubsection*{2.1.4.2 Richard Joyce's revolutionary fictionalism: moral discourse is hopelessly flawed}
According to Joyce, (a) moral language is flawed at its fundamental level (i.e. moral assesions cannot be of truth). But Joyce does not give up. Rather, (b) we should keep such a flawed language and use it \emph{in an improved way}.
Note that his fictionalism is not intended to descrive what we actually use but to suggest a better way for talking about morality.

\subsection*{2.1.5 Revisionism about truth}
Since Tarski, philosophers have been familiar to paradoxes witnessing that our concept `truth' is defective.

Scharp (2007, 2013) \emph{replaces} concepts in four steps (see ch.8 for detail).
  (1) Pre-revolution. Use concept and theory for explaining things.
  (2) Early revolution. Discover the concept is inconsistent.
  (3) Late revolution. Suggest new concepts.
  (4) Post-revolution. Replace the theory.


\subsection*{2.1.6 More on inconsistent or incoherent concepts: Weiner and van Inwagen}
Scharp further generalizes inconsistency: \emph{all} philosophical concepts are inconsistent (Scharp 2013).
A more modest claim is seen in Weiner (2009, for `knowledge') and van Iwagen (2008, `for freedom').

\subsection*{2.1.7 Engineering the concept of race}
Kwme Anthony Appiah defends \emph{eliminativism} about races, according to which there is \emph{no} such thing as race.
Appiah points out that our concept of race requires a problematic assumption called \emph{racialism}, which claims that we can divide humans into groups called races whose members share several aspects (e.g. biological, cultural).
But empirical studies have shown that racialism is false. So there is no race.

Alternatively, Haslanger's amelioration project \emph{improves} our concepts of race. She does not fight against the fact we in face have these concepts of races but advises to have a better version.

\subsection*{2.1.8 Leslie on generics and social prejudice}

Sarah Jane Leslie gives a different style of conceptual engineering.
Her argument cites empirical evidence supporting that to use certain linguistic expressions (called generics) causes cognitive errors and stress (prejudices).

\subsection*{2.1.9 Epistemology: `what is knowledge?' or `what should knowledge be?'}

Cappelen highlights the difference between \emph{descriptive} epistemology and \emph{revisionary} epistemology.
A mainstream (including Gettier and the following literatures, also Williamson's ``knowlege-first'' direction) has executed \emph{descriptive} tasks for knowledge. They ask ``what \emph{is} knowledge''.

In contrast, conceptual engineers have asked and ask: ``what \emph{should} our concept of knowlege be?'' as Haslanger does.

\subsection*{2.1.10 Carnap on nonsense}
The last example on the list is Caprnap on \emph{nonsense}, which exemplifies the most radical view on how \emph{deficient} concepts can be. Wittgenstein, Carnap and other logical positivists argue that many philosophical questions are \emph{meaningless}. Not only metaphysical terms but also any normative claim is nonsense according to Carnap.

Cappelen himself suggested a more particular version of  Carnappian view, summarized as \emph{Pocket Nonsense}.

\subsection*{2.1.11 Is all philosophy conceptual engineering?}
We have observed ten examples of conceptual engineering. Should philosophers always do conceptual enginnering?
Some, including Eklund would say yes in a normative form: it is a norm for (all) philosophers (following Chalmers).
Others such as Blackburn nods with a descriptive stance: philosophy as a matter of fact do conceptual engineering all the time.

\subsection*{2.1.12 Chalmers on conceptual pluralism and pointless verbal disputes}
Cappelen disucusses Chalmers' argument to appeal conceptual engineering.
According to Chalmers' conceptual pluralism, many concepts desrves equal attention and imortance. This attitude entails philosophical pluralism, in which we should abandon the project of describing the concepts expressed by our words.
Conceptual engineering can still conduct a constructive project: to identify ``all relevant concepts in the vicinity of a particular term and seeing to what use they can be put.'' (p. 23)

Chalmers thinks, due to our ignorance to conceptual pluralism and conceptual engineering, that most debates in philosophy has been ``wastes of time'' by spending time to ``pointless verbal disputes'' , disagreements in disguise (p. 23).
Conceptual engineering is the way to save this.

\subsection*{2.1.13 Brief historical digression: analytic philosophy, Strawson, Soames, and contemporary semantics and epistemology}
A history of philosophy can (or should) be written as a battle between descriptivists and revisionists.


\section*{2.2 Conceptual Engineering beyond Philosophy}
In Section II, examples of conceptual engineering are gathered from outside of philosophy.
Cappelen claims that conceptual  engineering happens ``whenever humans communicate using language''. (p. 27)

\subsection*{2.2.1 Conceptual engineering in law and psychiatry}
Lawers often ask what a legal sentence \emph{should} mean (and revise if needed). Psychiatrists have invested a lot to think about how terms should be defined and what should be in their extensions when they do \emph{revise} the DSM classifications.
\subsection*{2.2.2 Public controversies over `person', `marriage', `rape', and Biko on `black vs. non-white'}
Many public debates are quite conceptual engineering. They are about what out words should mean.
Cappelen cites examples found in Peter Ludlow's Living Words (Ludlow 2014) as instances of `meaning negotiation'.
The details of his theory will be descussed later in chapter 15.

Examples include:
`marriage' (whether gay marriage is a marriage), `rape' (whether unwanted sex within marriage is rape), and `person' (is a fetus a person?)

Cappelen disagrees with an interpretation which sees these debates as ``first-order'' ones. So construed, ``one view is right and the other is wrong -- the issue isn't settled by fixing concepts''.
Instead, Cappelen arugues that they are about our concept: ``what our words should mean, or what concepts those words should express''(p.28).

Cappelen offers an argument supporting this finding. The upshot is: many do not change their opinion just by saying ``it is what the word means''.

Many other words can be targets of conceptual engineering (his own; immigrant, refugee, expat; money, poverty; Ludlow's; doll, sandwich, journalist, relevant, organic, and athlete).
Cappelen reminds us that this book is more about  the general structure among these examples rather than detailed analysis for each.

\subsection*{2.2.3 Related phenomena: semantic drift and contextual negotiations}
Two linguistic phenomena resemble non-philosophical instances: gradual semantic drift and negotiation on how to fix the semantic values of context-sensitive terms in given contexts.

\subsubsection*{2.2.3.1 Sematnic drift and semantic plasticity: a continuous process of conceptual engineering?}
Gradual semantic drift happens when our words change their meaning/extension/intension through time.
For example, Dorr and Hawthorne explains how the word salad began to cover fruit salad. Dorr and Hawthorne suggest two models: \emph{patchwork} and \emph{plasticity}.

\subsubsection*{2.2.3.2 Contextual negotiation as a form of conceptual engineering?}
Consider indexicals for example.
Cappelen will argue agains the radical contextualism in chapter 15.


\section*{2.3 The Logical Space of Conceptual Engineering: A Taxonomy}
Cappelen moves to the fundamental structure observed throughout the examples above.
Cappelen (hesitadely) makes a tentative taxonomy of conceptual engineering. The taxonomy contains six main points, corresponding to sub-subsections.

\subsection*{2.3.1 The varieties ofconceptual deficienc}
\emph{What is the defect?}
\footnote{Comment by Endo. This part is the least organized part which confuses me. }
\subsection*{2.3.2 The varieties ofameliorative strategies}
\emph{How to rescue the defect?} Cappelen prepares three different strategies.
\subsubsection*{2.3.2.1 Improve the concept and keep the lexical item}
Do not change the list of words (i.e. \emph{lexical items})and improve meaning/extension/intension.
Some (see chapter 9) criticizes this as an incoherent plan.
But philosophers in examples all want to keep the same expression.
\subsubsection*{2.3.2.2 Improve the concept and change the lexical item}
Chalmers' paper ``Verbal Disputes'' (Chalmers 2011), for example, not only improve the meaning/extension/intension but also find a new expression (Chalmers uses subscript to the old expression).
\subsubsection*{2.3.2.3 Complete abandonment}
Give up such hopeless things. Like Canap did for metaphysics.

\subsection*{2.3.3 Intentional vs. unintentional}
\emph{Is the revision is intentional?}
Philosophers (part I) were cited as examples of intensional (self-aware) conceptual engineering. Non-philosophers (part II) often do conceptual engineering without the practicers do not describe as such.
\subsection*{2.3.4 Local vs. broad conceptual engineering}
\emph{The scope of the revision?}
\subsection*{2.3.5 Institutional vs. non-institutional}
\emph{Does the revision have institutional support?}
Non-philophical cases such as laws and psychiatry are often regulated and instituitionalized. Often backed up by other power (e.g. police for laws). Philosophers are more informal. They do not usually have such political power to support their conceptual engineering.
\subsection*{2.3.6 The kinds ofnorms involved}
\emph{What norms motivate the revision?}
Cappelen does not give an answer because he thinks there is no point of doing that.
\subsection*{2.3.7 Two things I haven't mentioned: holism and creating from scratch}
Cappelen mentions two topics related to and unsaid in the previous parts of this chapter: holism and creating from scratch.

\paragraph{Holism (and more).}
As for holism-molecularism-atomism discussion, their couterparts in conceptual engineering are as follows.

i) holsm: conceptual engineering affects the meaning of any other term
ii) molecularism: conceptual engineering affects the meaning of  only`neighborhood' concepts
iii) atomism: conceptual engineering affect the meaning only of the concept in question.

\paragraph{Creating from scratch.}
Cappelen explains why he does \emph{not} discuss creating concepts from scratch and concentrates on improving and replacing the existing concepts.

The first reason is descriptive (in a way). His examples -- what Cappelen calls `data points'-- concern only improvement.
So skipping creating the concept is no problem.

The second reason has something to do with his externalistic stance. He presupposes that ``the idea of baptisms and introductory definitions plays a role'' (p. 37). Non-creating cases would be enough to guess the normative facets of creating from scratch in case needed.

\section*{Comments by me.}
%----------------------------------------------------------------------------------------
%	BIBLIOGRAPHY
%----------------------------------------------------------------------------------------

%\renewcommand{\refname}{Reference} % Change the default bibliography title

\bibliography{Mendeley} % Input your bibliography file

%----------------------------------------------------------------------------------------

\end{document}
