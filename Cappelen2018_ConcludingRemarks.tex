%%%%%%%%%%%%%%%%%%%%%%%%%%%%%%%%%%%%%%%%%
% Article Notes
% LaTeX Template
% Version 1.0 (1/10/15)
%
% This template has been downloaded from:
% http://www.LaTeXTemplates.com
%
% Authors:
% Vel (vel@latextemplates.com)
% Christopher Eliot (christopher.eliot@hofstra.edu)
% Anthony Dardis (anthony.dardis@hofstra.edu)
%
% License:
% CC BY-NC-SA 3.0 (http://creativecommons.org/licenses/by-nc-sa/3.0/)
%
%%%%%%%%%%%%%%%%%%%%%%%%%%%%%%%%%%%%%%%%%

%----------------------------------------------------------------------------------------
%	PACKAGES AND OTHER DOCUMENT CONFIGURATIONS
%----------------------------------------------------------------------------------------

\documentclass[
10pt, % Default font size is 10pt, can alternatively be 11pt or 12pt
a4paper, % Alternatively letterpaper for US letter
twocolumn, % Alternatively onecolumn
landscape % Alternatively portrait
]{article}

%%%%%%%%%%%%%%%%%%%%%%%%%%%%%%%%%%%%%%%%%
% Article Notes
% Structure Specification File
% Version 1.0 (1/10/15)
%
% This file has been downloaded from:
% http://www.LaTeXTemplates.com
%
% Authors:
% Vel (vel@latextemplates.com)
% Christopher Eliot (christopher.eliot@hofstra.edu)
% Anthony Dardis (anthony.dardis@hofstra.edu)
%
% License:
% CC BY-NC-SA 3.0 (http://creativecommons.org/licenses/by-nc-sa/3.0/)
%
%%%%%%%%%%%%%%%%%%%%%%%%%%%%%%%%%%%%%%%%%

%----------------------------------------------------------------------------------------
%	REQUIRED PACKAGES
%----------------------------------------------------------------------------------------

\usepackage[includeheadfoot,columnsep=2cm, left=1in, right=1in, top=.5in, bottom=.5in]{geometry} % Margins

\usepackage[T1]{fontenc} % For international characters
\usepackage{XCharter} % XCharter as the main font

\usepackage{natbib} % Use natbib to manage the reference
\bibliographystyle{apalike} % Citation style

\usepackage[english]{babel} % Use english by default

%----------------------------------------------------------------------------------------
%	CUSTOM COMMANDS
%----------------------------------------------------------------------------------------

\newcommand{\articletitle}[1]{\renewcommand{\articletitle}{#1}} % Define a command for storing the article title
\newcommand{\articlecitation}[1]{\renewcommand{\articlecitation}{#1}} % Define a command for storing the article citation
\newcommand{\doctitle}{\articlecitation\ --- ``\articletitle''} % Define a command to store the article information as it will appear in the title and header

\newcommand{\datenotesstarted}[1]{\renewcommand{\datenotesstarted}{#1}} % Define a command to store the date when notes were first made
\newcommand{\docdate}[1]{\renewcommand{\docdate}{#1}} % Define a command to store the date line in the title

\newcommand{\docauthor}[1]{\renewcommand{\docauthor}{#1}} % Define a command for storing the article notes author

% Define a command for the structure of the document title
\newcommand{\printtitle}{
\begin{center}
\textbf{\Large{\doctitle}}

\docdate

\docauthor
\end{center}
}

%----------------------------------------------------------------------------------------
%	STRUCTURE MODIFICATIONS
%----------------------------------------------------------------------------------------

\setlength{\parskip}{3pt} % Slightly increase spacing between paragraphs

% Uncomment to center section titles
%\usepackage{sectsty}
%\sectionfont{\centering}

% Uncomment for Roman numerals for section numbers
%\renewcommand\thesection{\Roman{section}}
 % Input the file specifying the document layout and structure
\usepackage{hyperref}


%----------------------------------------------------------------------------------------
%	ARTICLE INFORMATION
%----------------------------------------------------------------------------------------

\articletitle{Summary of Concluding Remarks and the Limits of the Intellect} % The title of the article
\articlecitation{\cite{Cappelen2018FixingLanguage}} % The BibTeX citation key from your bibliography

\datenotesstarted{January 13, 2019} % The date when these notes were first made
\docdate{\datenotesstarted; rev. \today} % The date when the notes were lasted updated (automatically the current date)
\docauthor{Summarized and Commented by Shimpei Endo} % Your name

%----------------------------------------------------------------------------------------

\begin{document}

\pagestyle{myheadings} % Use custom headers
\markright{\doctitle} % Place the article information into the header

%----------------------------------------------------------------------------------------
%	PRINT ARTICLE INFORMATION
%----------------------------------------------------------------------------------------

\thispagestyle{plain} % Plain formatting on the first page

\printtitle % Print the title

%----------------------------------------------------------------------------------------
%	ARTICLE NOTES
%----------------------------------------------------------------------------------------

\section*{In a nutshell... }
This concluding chapter justifies the naming of ``conceptual engineering'' and specifies his seemingly pessimistic but still progressive standpoint.

\noindent \textbf{Keywords:} conceptual engineering

\section*{(Contents of this chapter)}
\paragraph{`Conceptual enginnering' is not a perfect name.}
Cappelen admits that `conceptual engineering' is not the best way to label what he has meant and argued. So far, Cappelen has provided ``a theory of conceptual enginnering without concepts and without engineering''
[p. 199].
The earlier part of this chapter offers three justifications for this bad naming.

\paragraph{Reason 1: For philosophers at work to notice. }
The main reason is to capture ``the self-image'' of philosophers in front lines (see Part I, listed as ``data points'').
The name of conceptual enginnering is ``familiar to the people and the traditions that the book theorizes about'' [p. 199].

\paragraph{Reason 2: To revise what conceptual engineers actually do. }
Relating to the first point,
Cappelen intends to revise the self-understanding of philosophers.
They, reasonably, misunderstand what they are actually doing.

\begin{quote}
  In particular, people who spend big chunks of their lives thinking, writing, and talking tend to think that those activities are important and have significant impact. For the most part that is not so.
\end{quote}

\paragraph{Reason 3: To call competitors.}
Cappelen also designed the name to call attention for other competing frameworks.
Cappelen even digests their tasks of futhre conceptual engineering in the following two-fold manner:

\begin{enumerate}
  \item Identify the conceptual core--the relevant (proper) subset of entities in your setting.
  \item Show how we can engineer these things.
\end{enumerate}

\paragraph{Pessimistic?}

\paragraph{}
Symphasizing with this concern, Cappelen ....

\paragraph{The last paragraph.}
\begin{quotation}
  That said, the analogy goes only so far. The limitations on our ability to improve our representational devices are perhaps more painful and cut deeper than the other limitations I've mentioned. We are animals who pride ourselves on our rationality. The ability to think and represent is at the core ofthat rationality. That ability enables us to recognize both that our own representational devices are defective and that there isn't much we can do about it. We can observe these defects, describe them, reflect on them, and think of ameliorative strategies. But careful thinking also reveals that such reflection is ineffective. Amelioration might happen, but if it does, it has little to do with our intentional efforts. Our intellect can diagnose itself, figure out a cure, but is impotent when it comes to doing anything. Emphasizing this highlights an important limitation on human rationality and intellect.
\end{quotation}

\section*{Comments by Endo}
%----------------------------------------------------------------------------------------
%	BIBLIOGRAPHY
%----------------------------------------------------------------------------------------

%\renewcommand{\refname}{Reference} % Change the default bibliography title

\bibliography{Mendeley} % Input your bibliography file

%----------------------------------------------------------------------------------------

\end{document}
